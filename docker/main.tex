\documentclass{beamer}
 
\usepackage[utf8]{inputenc}
\usepackage{graphicx} 
\graphicspath{{img/}}

\usetheme{Copenhagen} 
\title{Containers in a nutshell}
\author{Simone Lombardi}
\institute{HCSSLUG | hcsslug.org}
\date{27 Ottobre 2017}
%\logo{\includegraphics[height=0.5cm]{hcsslug.png}}  
\begin{document}
 
\frame{\titlepage}

\begin{frame}
    \frametitle{Introduzione}
    \begin{center}
        Cosa sono i container? 
    \end{center}
\end{frame}

% --------------------------------

\begin{frame}
    \frametitle{Introduzione}
        I container sono un environment di esecuzione self-contained: hanno a disposizione le loro risorse 
        e condividono con il sistema host il kernel.
\end{frame}

% --------------------------------
     
\begin{frame}
    \frametitle{Vantaggi dei container}
    I container portano numerosi vantaggi: 
    \begin{itemize}
        \item<1-> Portabilit\`a
    \item<2-> Sharing 
        \item<3-> Velocit\`a di deploy 
        \item<4-> Footprint
        \item<5-> Manutenzione
    \end{itemize}
\end{frame}

% --------------------------------

\begin{frame}
    \frametitle{Qualche nome...}
    I progetti pi\`u famosi sono: 
    \begin{itemize}
        \item<1-> LXC
        \item<2-> LXD
        \item<3-> Docker
    \end{itemize}
\end{frame}

% --------------------------------

\begin{frame}
    \frametitle{Docker in a nutshell}
    \begin{center}    
        \includegraphics{{docker_logo.png}}
    \end{center}
\end{frame}

% --------------------------------

\begin{frame}
    \frametitle{Docker in a nutshell}
    Docker \`e un progetto FOSS che permette di automatizzare il deploy di applicazioni in un Linux Container, mettendo a disposizione molti tool per rendere 
    le operazioni di gestione rapide e intuitive. 
\end{frame}

% --------------------------------

\begin{frame}
    \frametitle{Come funziona Docker?} 
    \begin{itemize}
        \item<1-> \textbf{cgroups}
        \item<2-> \textbf{namespaces}
        \item<3-> \textbf{OverlayFS}
        \item<4-> \textbf{libcontainer}
    \end{itemize}
\end{frame}

% --------------------------------

\begin{frame}
    \frametitle{Perch\`e Docker?}
    Provvede all'isolamento degli applicativi, inoltre l'effort per configurare un container `base` \`e minimo. 
    \begin{itemize}
        \item<1-> Creazione e distruzione dei container molto rapida. 
        \item<2-> Velocit\`a e leggerezza
        \item<3-> Risorse condivise col sistema Host
    \end{itemize}
\end{frame}

% ---------------------------------

\begin{frame}
    \frametitle{Svantaggi di Docker}
    \begin{center}
        \textbf{Confronto con le VM}
    \end{center}
    \begin{itemize}
        \item<1-> Le VM possono essere `spostate` mentre sono in esecuzione. 
        \item<2-> I Container \textbf{NON} rimpiazzano le VM in ogni caso.
        \item<3-> Valutare sempre gli USE Case!
    \end{itemize}
\end{frame}

% --------------------------------

\begin{frame}
    \frametitle{Un piccolo assaggio}
    Docker pu\`o buildare immagini leggendo un set di istruzioni da un file, chiamato \texbf{Dockerfile}.
    In questo file possiamo definire il comportamento del container, usando varie features implementate:
    \begin{itemize}
        \item<1-> Volumes
        \item<2-> Expose
    \end{itemize}
\end{frame}

\begin{frame}
    \frametitle{Dockerfile}
    \begin{center}
        \includegraphics[width=10cm,height=10cm,keepaspectratio]{dockerfile.png}
        \\ \textbf{In figura:} Esempio di Dockerfile
    \end{center}
\end{frame}

% --------------------------------

\begin{frame}
    \frametitle{Container multipli}
    Vi sono a disposizione innumerevoli tool per gestire applicativi multi-container contemporaneamente, \`e qui che entra in gioco \textbf{Docker Compose}.
\end{frame}

% --------------------------------

\begin{frame}
    \frametitle{Docker Compose}
    Compose \`e un tool per definire e avviare pi\`u container, definiti in un file .yaml, ha molte features interessanti:
    \begin{itemize}
        \item<1-> Environments isolati su un singolo host
        \item<2-> I dati dei volumi sono preservati
        \item<3-> \`E possibile ricreare solo i container modificati
        \item<4-> Sostituzioni Variabili
    \end{itemize}
\end{frame}

\begin{frame}
    \frametitle{Conclusione}
    Docker \`e uno strumento molto potente che permette davvero di organizzare il proprio workflow in maniera pulita ed ottimizzata.
    L'unico limite \`e la fantasia.
\end{frame}


\begin{frame}
    \frametitle{Fine}
    \begin{center}
        \textbf{Grazie a tutti per l'attenzione :D}
    \end{center}
\end{frame}
\end{document}

